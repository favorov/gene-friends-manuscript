\documentclass{llncs}
%\usepackage{fancyhdr}
\pagestyle{plain}
%\pagestyle{headings}
\usepackage{standalone}
\usepackage{graphicx} 
\usepackage{comment}
%\usepackage{xcolor}
\usepackage{epigraph}
\usepackage{amsmath}
\usepackage{amsfonts}
\usepackage{amssymb}
\usepackage{cite}
%\usepackage{natbib}
%\usepackage[title]{appendix}
\usepackage{caption}
\usepackage{subcaption}
\usepackage{latexcolors}
\usepackage{import}
\usepackage{tikz}
\usetikzlibrary{positioning}
\usetikzlibrary {arrows.meta}



\begin{document}

%


%\title{Telling the story of best friends}
%
\title{Best friends test: differential importance statistical test reveals hidden specific relations}
%
\titlerunning{Best friends test}  
% abbreviated title (for running head)
%                                     also used for the TOC unless
%                                     \toctitle is used
%
\author{
Alexandra Suvorikova \inst{1} \and Vasily Ramensky \inst{2,3,4} \and Vera Mukhina \inst{5,6} \and Ludmila Danilova\inst{8} \and Andrey Mironov \inst{4,7} \and Alexander Favorov\inst{6,8}}
%
\authorrunning{A. Suvorikova et al.} % abbreviated author list (for running head)
%
%%%% list of authors for the TOC (use if author list has to be modified)
\tocauthor{Alexandra Suvorikova, Vasily Ramensky, Vera Mukhina, Ludmila Danilova, Andrey Mironov, and Alexander Favorov}
%
\institute{
Weierstrass Institute, Berlin, Germany, 
\email{suvorikova@wias-berlin.de}
\and
MSU Institute for Artificial Intelligence, \\Lomonosov Moscow State University,  Moscow, 119992, RF
\and
National Medical Research Center for Therapy and Preventive Medicine of the Ministry of Healthcare of Russian Federation,  Moscow, 101990, RF
\and
Department of Bioengineering and Bioinformatics, \\Lomonosov Moscow State University,  Moscow, 119992, RF
\and
University of Maryland School of Medicine,  Baltimore, MD 21205, USA
\and
Johns Hopkins University School of Medicine, \\ Baltimore, MD 21205, USA, \email{favorov@sensi.org}
\and
The Institute for Information Transmission Problems,  Moscow, 127051, RF
\and
Vavilov Institute of General Genetics, RAS,  Moscow, 119333, RF
}

\maketitle              % typeset the title of the contribution

\renewcommand{\tag}{tag}
\newcommand{\collection}{collection}
\newcommand{\T}{T}
\newcommand{\C}{C}
\newcommand{\tl}{t}
\newcommand{\cl}{c}
\newcommand{\test}[1]{\textbf{\textit{#1}}}
%\setlength\epigraphwidth{.8\textwidth}
\setlength\epigraphrule{0pt}

\epigraph{SI AUGUSTUS CERNATUR, CERNANTUR QOUQUE AMICI}

\vspace{-0.2in}

\begin{abstract}
We propose a novel approach to select hidden specific relations in a dataset, which a bipartite graph can represent. 
To describe the hidden specific relations in general, we use a concept of \textit{friendship}. The goal is to detect objects in the data that are more important to particular others, than to the rest of the dataset, i.e. to detect friends. To this end, we introduce two statistical tests, the \test{best friend test} and the \test{friends test}. Both tests are based on the double ranking of the entries in the block of the weighted adjacency matrix of the bipartite graph. This model fits many practical problems, such as gene expression regulation by a set of transcription factors, etc. The method is available as an \textsf{R} package at \url{https://github.com/favorov/best.friends}.
\end{abstract}


\keywords{weighted bipartite graph, hidden relations, rank statistics, feature selection, clustering, knowledge transfer, specific gene regulation, pattern marker}

\section{Introduction: what it means to be a friend}

% There is a simple intuition of what it means to be a friend. Augustus is more important to his friend than to other people. And if we see 
%Augustus, then we also infer that we will see friend(s) of Augustus. 
There is a simple intuition of what it means to be a friend.  
The friends of Augustus try to get together with him as often as possible. And if we meet August somewhere, we will likely meet his friends too.
In the following, we numerically formulate this concept as a double-ranking procedure and a statistical test that validates the ranking result. This approach is multi-purpose and detects hidden specific interactions.
% In the following, we numerically formulate this concept as a double ranking procedure and a statistical test that validates its result to create a multi-purpose numeric method that detects hidden specific interactions.

A motivating example comes from the decomposition of omics data 
\cite{fertig_cogaps_2010, stein-obrien_enter_2018}. One of the results is a matrix of weights of experimentally measured features (e.g. gene expressions) in complex biological processes. If a feature has a substantial weight in only one process, it is referred to as a marker \cite{stein-obrien_patternmarkers_2017}. 
The presence of a marker indicates the activity of the marked process.
The goal is to find all the marker features and the corresponding processes.

Of note, neither a noise feature nor a feature with large weights in all processes should be identified as a marker. To achieve this, we redefine markers as features significantly more important for a particular process than for other processes. The friend is a complementary concept to this definition of a marker. If a process is a friend to a feature, the feature is a marker to the process, and vice versa. 
Introducing terms for both friends and markers, we emphasize that the friendship relation is asymmetric. We will explain the idea later.

\textcolor{navyblue}{The approach we introduce here applies to numerous problems. So, to generalize the setting, we will refer to features as \textit{{\tag}s}, to processes as \textit{{\collection}s}, and to weights as \textit{attention}. A bipartite graph suits well to picture it (fig.\ref{fig:nice_name}).  The attention is the weight of an edge between two edges, a collection and a tag.}

%The bipartite graph suits well to the above-mentioned setup. \textcolor{red}{The approach we will introduce applies to many setups}. So, to generalize the setting, we will refer to features as \textit{{\tag}s}, to processes as \textit{{\collection}s}, and to weights as \textit{attention}. Regarding graphs, attention is the weight of an edge between a tag and a collection. Fig.\ref{fig:nice_name} illustrates the setting.

\begin{figure}
    \centering
    \import{}{bipartite}
    \caption{Bipartite graph presenting tag-collection model. $A(t,c)$ is a collection's attention to a tag. The arrows depict nonzero $A(t,c)$ .}
    \label{fig:nice_name}
\end{figure}

Let a set of collections be $C = \{c_1, \dots, c_k\}$, and a set of tags be $T = \{t_1, \dots, t_n\}$.
We set attention $a_{ij}$ to be the weight of an edge between $t_i \in T$ and $c_j \in C$.
Let's agree that the greater the $a_{ij}$, the higher the attention is. Naturally, the absence of attention corresponds to $a_{ij} = 0$. 

Values $a_{ij}$ are stored in a matrix $\mathcal{A}$. One may think of $A$ as a non-diagonal block of a weighted adjacency matrix of the corresponding bipartite graph. Each row $\mathcal{A}_{i:}$ corresponds to attention that a tag $t_i$ receives from all collections; each column $\mathcal{A}_{:j}$ corresponds to attention from the collection $c_j$ to all tags.

This model applies to many setups (see Tab.\ref{tab:examples} for examples).

\begin{table}[h!]
\centering
\caption{Tag-collection model deployment examples.}
\label{tab:examples}
\begin{tabular}{c|c|c|c}
\textbf{Example} & \textbf{Tag $t_i$} & \textbf{Collection $c_j$} & \textbf{Attention $a_{ij}$} \\ 
\hline

\begin{tabular}[c]{@{}c@{}}Search engine\end{tabular} & Query       & \begin{tabular}[c]{@{}c@{}} Search result \end{tabular}         & \begin{tabular}[c]{@{}c@{}}Relevance of\\ search output\\\end{tabular} \\ \hline

\begin{tabular}[c]{@{}c@{}}Gene expression regulation\\  by transcription factors\end{tabular}     & Gene             & \begin{tabular}[c]{@{}c@{}}Genes under regulation \\ by the transcription factor \end{tabular}     & \begin{tabular}[c]{@{}c@{}}Strength of \\ regulation\end{tabular}       \\ \hline

\begin{tabular}[c]{@{}c@{}}Transcription\\ decomposition\end{tabular} & Transcript       & \begin{tabular}[c]{@{}c@{}}Complex biological process\end{tabular}         & \begin{tabular}[c]{@{}c@{}}Transcript's weights \\ in process\end{tabular} \\ \hline

\begin{tabular}[c]{@{}c@{}}Transcription \\ correlation\end{tabular} & Transcript             & \begin{tabular}[c]{@{}c@{}}Another transcript\end{tabular} & \begin{tabular}[c]{@{}c@{}}Correlation of transcription values\\ measured in different experiments\end{tabular}    \\ \hline
Fuzzy clustering                                                      & Object           & Cluster                                                                  & \begin{tabular}[c]{@{}c@{}}Object weight \\ in cluster\end{tabular}     \\ \hline


\begin{tabular}[c]{@{}c@{}}Weighted graph\end{tabular} & Vertex       & \begin{tabular}[c]{@{}c@{}} Another vertex \end{tabular}         & \begin{tabular}[c]{@{}c@{}}Weight of edge between\\ collection and tag \end{tabular} \\ \hline

\end{tabular}
\end{table}

A tag is more important to its friends than to other collections. In the following, we express the importance by ranking.  

The ranking can be applied in a na\"ive way: the higher the attention of a collection to a tag is, the more friendly the collection is to the tag. But this approach cannot distinguish between the tags that are specific markers and those that are important for all collections (e.g., so-called network hubs).

We suggest using double-ranking. It points to the most friendly collection for a given tag. 

First, for each collection $c_j$ we decreasingly rank the attention values $a_{1j}, \dots, a_{nj}$ and get the corresponding ranks $r_{1j}, \dots, r_{nj}$.  The lower $r_{ij}$ is, the higher the importance of the tag $t_i$ for the collection $c_j$. See Section~\ref{sec:method} for details.

Second, we fix a tag $t_i$ and rank $r_{i1}, \dots, r_{ik}$. In other words, we rank all collections by this tag's importance for them.
This step numerically expresses the friendship concept. Say the collection $c_j$ has the highest rank $r_{ij}$ among $r_{i1}, \dots, r_{ik}$. Thus, the importance of $t_i$ to $c_j$ is higher than to other collections. Thus, the collection $c_j$ can be $t_i$'s best friend.

Having the highest rank is necessary but insufficient to be the real best friend. We can illustrate this as follows. Even in a randomly generated attention matrix $\mathcal{A}$, the double ranking will bring up a collection for each tag. So, we suggest a novel statistical test that assesses the reliability of friendship in this setup.

Friendship in this notation is a property of the whole matrix $\mathcal{A}$.
Under $H_0$, which suggests no friendship at all, we assume the attention values from a collection $c_j$ to all tags are independent and identically distributed.
This distribution can vary from collection to collection. Moreover, all the $n \times k$ attention values $a_{ij}$ from collections to tags are independent.

We suggest a test checking whether $c_j$ is really the best friend of $t_i$ or $c_j$ appears at the top of the ranking by chance (i.e. according to $H_0$). The test is $p$-value based and compares the ordered ranks for $t_i$ in different collections. We name it the \test{best friend test}.

%\textcolor{purple}{Looking for a possible best friend and statistically assessing the friendship, we start with a data set represented by a bipartite graph. Its vertices are tags and collections and its edges are attention from collection to tag. To start with, we choose a tag and then we look for a collection. As we see, the procedure is asymmetric by design. If it succeeds, we refer to the collection $c_j$ as the best friend of the tag $t_i$; $t_i$ selects $c_j$ and we refer to $t_i$ as a marker for $c_j$. The term is consistent with the pattern marker term \cite{stein-obrien_patternmarkers_2017}.}
%\textcolor{red}{See the commented comment in Russian below}
%Перечитала текст и не поняла что это за покемон и зачем он здесь????
%ЭЭЭЭЭ фиг знает что это, наверное, из-под коммента выпрыгнуло без разрешения
%надо проверить, не пригодится ли это в дискуссию - и всё

So far, we consider only one possible best friend $c_j$ per tag $t_i$. The procedure naturally expands to the case when the tag selects/marks multiple collections, thus having more than one friend. In other words, the \test{best friend test} expands to the \test{friends test}.

The tag splits the set of all collections into two subsets, e.g. the friends of the tag and all the others. The split occurs by a chosen threshold for ordered ranks. The \test{friends test} estimates the statistical significance of this split.

Both tests run for only one tag of interest. Since we usually have no apriori preferences for a tag, we can run the tests for all the tags simultaneously. In this case, we should consider the multiplicity correction. 

\section{Method}
\label{sec:method}


We recall that the attention matrix $\mathcal{A}$ is the only input for the statistical test. Each its element $a_{ij}$ is the value of attention that a collection $c_j \in C$ pays to a tag $t_i \in T$
\[
\mathcal{A} = \begin{pmatrix}
a_{11} & a_{12} & \dots & a_{1k} \\
       &\cdots & \cdots &  \\
a_{n1} & a_{n2} & \dots & a_{nk}
\end{pmatrix}.
\]
Its $i$-th row corresponds to tag $t_i$, and we denote it as $\mathcal{A}_{i:} = (a_{i1}, \dots, a_{ik})$. Its $j$-th  corresponds to collection $c_j$, and we denote it as $\mathcal{A}_{:j} =(a_{1j}, \dots a_{nj})'$ (with $'$ being transposition).

%from intro We express how much a collection $c_j$ cares about a tag $t_i$ in terms of the rank of $a_{ij}$ in column $\mathcal{A}_{:j}$. So, if a collection $c_j$ cares about a tag $t_i$ more than other collections, the rank of $a_{ij}$ in $\mathcal{A}_{:j}$ is higher than the rank of $a_{il}$ in $\mathcal{A}_{:l}$ for all $l \neq j$. 

We do the following procedure to identify the collection, which is the putative best friend (or, in other words, the most friendly collection) for each tag.

For each collection $c_j \in C$, we decreasingly rank the elements inside $\mathcal{A}_{:j}$. Thus, for each $a_{ij} \in \mathcal{A}_{:j}$, we get the ordinal number $r_{ij}:=\text{rank}\left(a_{ij}|\text{inside}~\mathcal{A}_{:j}\right)$. We resolve the ties by the mean rank of the tied elements. 

Let's denote the rank matrix as 
\begin{equation}
\label{def:R}
\mathcal{R} = \begin{pmatrix}
r_{11} & r_{12} & \dots & r_{1k} \\
       &\cdots & \cdots &  \\
r_{n1} & r_{n2} & \dots & r_{nk}
\end{pmatrix}, 
\quad
r_{ij} =\text{rank}\left(a_{ij}|\text{inside}~\mathcal{A}_{:j}\right).
\end{equation}
We denote its $i$-th row as $\mathcal{R}_{i:} = (r_{i1}, \dots, r_{ik})$.

To quantitatively express friendliness, we order the attention ranks to the same tag $t_i$ from different collections. We decreasingly rank the elements in $\mathcal{R}_{i:}$ and the minimal element
corresponds to the collection that is the putative best friend of $t_i$.

Let's denote the collection-index (the second index) of the smallest entry in $\mathcal{R}_{i:}$ as $\sigma_i(1)$, the collection-index of the second smallest entry as ${\sigma_i(2)}$, etc. The corresponding collections are $c_{\sigma_{i}(1)}$, $c_{\sigma_{i}(2)}$, etc. 

So the putative best friend for the tag $t_i$ is the collection $c_{\sigma_{i}(1)}$.

\subsection{The best friend test}
\label{sec:best_friend_test}

For a collection to be a tag's best friend, it is \textit{necessary} to be the putative best friend, but it is \textit{not enough}. Indeed, in any ranking, there is the first element. We aim to statistically estimate whether the most friendly tag is the most friendly by chance. 

We recall that under $H_0$ all $a_{ij}$ are independent and all $a_{ij} \in \mathcal{A}_{:j}$ are i.i.d. So, by the construction, under $H_0$ all $r_{i\sigma(j)}$ are independent and uniformly distributed.
Thus, under $H_0$, each $\mathcal{R}_{i:}$ entries are independently uniformly distributed. 

For a tag $t_i$, let's order the elements of $\mathcal{R}_{i:}$. The smallest entry in $\mathcal{R}_{i:}$ is $r_{i\sigma_i(1)}$, the second smallest entry is $r_{i\sigma_i(2)}$, etc. For simplicity, we do not write the secondary $i$ index that is the same by construction as the first $i$. The resulting matrix is 
\begin{equation}
\label{def:R_sigma}
\mathcal{R}_{\sigma} = \begin{pmatrix}
r_{1\sigma(1)} & r_{1\sigma(2)} & \dots & r_{1\sigma(k)} \\
       &\cdots & \cdots &  \\
r_{n\sigma(1)} & r_{n\sigma(2)} & \dots & r_{n\sigma(k)}.
\end{pmatrix}
\end{equation}

In this notation $c_{\sigma_i(1)}$ is the putative best friend for the tag $t_i$, and $r_{i\sigma(1)}$ is the rank of $t_i$ in $\mathcal{R}_{i:}$. 

To test whether the putative best friend for the tag $t_i$ ($c_{\sigma_i(1)}$) is really the best friend, we use the observed (i.e. calculated for given $\mathcal{A}$) difference:
\begin{equation}
\label{def:u_1}
u_1(t_i) = \frac{r_{i\sigma(1)} -  r_{i\sigma(2)}}{n}.
\end{equation}
The $n$-normalization is technical. We explain it in more detail in Section \ref{sec:theory}.

We note that $u_1(t_i)$ is a realization of a random variable $U$ with the distribution known under $H_0$.

% We note that $u_1(t_i)$ is an observed value/\textcolor{red}{a realization} of a random variable $U$. Moreover, under $H_0$ the distribution of $U$ is known. We write it explicitly in Section \ref{sec:theory}. 

To test $H_0$ for any $t_i$ and its putative best friend $c_{\sigma_{i}(1)}$, we use the $p$-value,
%use $p$-value of the observed $u_1(t_i)$ in the distribution $U$,
\[
p = P\left(U \ge u_1(t_i)~|~H_0\right). 
\]

If $p$-value is small enough, we reject the null and claim that the friendliness of the collection $c_{\sigma_{i}(1)}$ is unlikely to observe by random, and so we refer to it as the best friend of $t_i$. In this case, $t_i$ is a marker of its best friend collection $c_{\sigma_{i}(1)}$.

\subsection{The friends test}
\label{sec:friends_test}

In some cases, a tag has several collections that are almost equally friendly to it. 
For example, a gene (tag) has a high load in two patterns (two collections), and all other genes are low in both collections. The tag (let it be $t_i$) is a marker for two collections $c_{\sigma_{i}(1)}$ and $c_{\sigma_{i}(2)}$. However, the best friend statistics for $t_i$ cannot find either of the two. 
Indeed, $c_{\sigma_{i}(1)}$ is better than $c_{\sigma_{i}(2)}$ just by chance and $H_0$ is correctly not rejected by the best friend test.

Still, it is possible that there are two consecutive collections $c_{\sigma_i(l)}$ and $c_{\sigma_i(l+1)}$, and the gap between these is statistically significant.

We denote the first $l$ collections that are most friendly as $F_{i}(l) = \left\{ c_{\sigma_i(1)} \dots c_{\sigma_i(l)} \right\}$.
We aim to check whether $F_{i}(l)$ is really a set of friends of the tag $t_i$. Numerically, it means that the gap 
\begin{equation}
\label{def:u_l}
u_{l}(t_i) = \frac{r_{i\sigma(l+1)} - r_{i\sigma(l)}}{n}
\end{equation}
is too large to be observed by chance if the null hypothesis $H_0$ holds. We note that $H_0$ is the same as in the \test{best friend test} (see Section \ref{sec:best_friend_test}).
Moreover, \test{best friend test} is a particular case of
this test---we will refer to it as \test{friends test}---with $l = 1$.

% The \test{friends test} possibly rejects $H_0$
% for the decomposition of $C$ into two group, $ F_{i}(l)$ and all the other collections.

Section \ref{sec:theory} proves that for any $i$ and $l$, $u_l(t_i)$ is a realization of $U$ introduced  in Section~\ref{sec:best_friend_test}.
By analogy with the \test{best friend test}, we asses $p$-value for $u_{l}(t_i)$,
%for the pair of $t_i$ and the population $l$ of the subset of collections $ \left\{ c_{\sigma_i(1)} \dots c_{\sigma_i(l)} \right\},$ that are putative friends between using $u_{l}(t_i)$,
\[
p = P\left(U \ge u_l(t_i)~|~H_0\right). 
\]
Again, we reject the null if the $p$-value is small enough. So, $F_{i}(l)$ are friends of $t_i$, and $t_i$ is their marker.


\subsection{Calculation of $p$-value}
\label{sec:theory}
The only data structure we use in the test is the matrix $\mathcal{R}$ defined in \eqref{def:R}.  
Its elements $r_{ij}$ are ranks of attention to $t_i$ paid by $c_j$.
We correct $r_{ij}$ for continuity and redefine them as normalized ranks,
\begin{equation}
\label{def:correction}
r_{ij} := \frac{r_{ij} - \frac{1}{2}}{n}.
\end{equation}

Of note, mapping \eqref{def:correction} coincides with the semi-discreet optimal transportation (under quadratic Euclidean cost) of a uniform distribution on an equally-spaced grid of integers to the continuous uniform distribution on $[0, 1]$ (see \cite{Solomon2018OptimalTO}).

To construct the test for a tag $t_i$, we consider the $i$-th row of $\mathcal{R}$, $\mathcal{R}_{i:} = (r_{i1}, \dots, r_{ik})$. For simplicity, we omit the first index $i$ and write
\[
r := (r_{1}, \dots, r_{k}).
\]
Under $H_0$, the vector $r$ is uniformly distributed on a $k$-dimensional cube $[0, 1]^{k}$.

By ordering $r$ we get
\[
r_{\sigma} = (r_{\sigma(1)}, \dots, r_{\sigma(k)}), 
\quad
r_{\sigma(1)} \leq r_{\sigma(2)} \leq \dots \leq r_{\sigma(k)}.
\]
We use the elements of $r_{\sigma}$ for the tests (see \eqref{def:u_1} and \eqref{def:u_l}). For simplicity of notations we denote $r_{\sigma(j)}$ as $u_j$ for all $j$,
\begin{equation}
\label{def:u}
    r_{\sigma} := u = (u_1, \dots, u_k).
\end{equation}


Vector $u$ takes values in $k$-dimensional convex polytope $P_k$ that is 
an intersection of $k$--dimensional cube $[0, 1]^{k}$ and 
$k-1$ half-spaces, which are defined by linear constraints $u_1 \le u_2$, $u_2 \le u_3$ etc.
Fig.\ref{fig:polytop} depicts $P_3$. 
\begin{figure}
     \centering
     \begin{subfigure}[b]{0.3\textwidth}
         \centering 
         \scalebox{.4}{\import{}{cube-volume}}
         \caption{Support of $u$.
         \\\hspace{\textwidth} 
        }
         \label{fig:polytop}
     \end{subfigure}
     \begin{subfigure}[b]{0.3\textwidth}
         \centering 
         \scalebox{.4}{\import{}{cube-w_1}}
         \caption{Support of $u$, 
         \\\hspace{\textwidth}
         $u_2 - u_1 \ge w$.}
         \label{fig:polytop1}
     \end{subfigure}
     \begin{subfigure}[b]{0.3\textwidth}
         \centering 
         \scalebox{.4}{\import{}{cube-w_2}}
         \caption{Support of $u$ \\\hspace{\textwidth}$u_3 - u_2 \ge w$.}
         \label{fig:polytop2}
     \end{subfigure}
    \caption{Polytopes $P_3$. The green area is the support of $u$, the dark-green area is the support of $u$ with additional coordinate constraints.}
\end{figure}

The volume $V_k$ of the $P_k$ is well-known,
\begin{eqnarray*}
V_k = &\displaystyle \int\limits_0^1\int\limits_{u_1}^1\int\limits_{u_2}^1\int\limits_{u_3}^1...\int\limits_{u_{k-1}}^1 du_k....du_4 du_3 du_2 du_1 =  \frac{1}{k!}~.
\end{eqnarray*}
% see \eqref{eq:volume} for inference.
% \textcolor{blue}{\textit{This estimation is well known and can be find in textbooks}}

Now we construct the random variable $U$ mentioned in Sections \ref{sec:best_friend_test} and \ref{sec:friends_test}. Given some fixed index $l\leq k-1$, we define
\[
U_{l} := u_{l+1} - u_{l}. 
\]
To estimate its $p$-value, we impose an additional restriction
\[
U_{l} \ge w, \quad w \in [0, 1].
\]

The vectors $u$ satisfying this restriction take values in a smaller polytope $P^{l}_{k}(w)$ that is an intersection of $P_{k}$ and a half-space represented by the restriction.
Fig.\ref{fig:polytop1} and Fig.\ref{fig:polytop2} depict $P^{1}_{3}(w)$, $P^{2}_{3}(w)$, respectively. 

The volume of $P^{l}_{k}(w)$ is 
\begin{equation}
V_{k}(w) = \frac{(1-w)^k}{k!},
\end{equation}
see \eqref{eq:p_1} and \eqref{eq:p_k} for the inference. We omit the $l$ index for the volume because it does not depend on $l$. The intuition is the following: the smaller polytope $P^{l}_{k}(w)$ is geometrically similar to $P_{k}$ with the scaling factor $(1-w)^k$. 

Finally, we recall that $u = r_{\sigma}$ (see \eqref{def:u}). Thus, under $H_0$ the probability for random vector $r_{\sigma}$ to satisfy the condition $U_{l} \ge w$ is the same for all $l\leq k-1$; it is equal to the ratio of $V_{k}(w)$ and $V_k$,
\begin{equation}
\label{eq:pw}
    p(w) = (1-w)^k.
\end{equation}



Since all $U_l$ have the same distribution, we use $U$-notation for $p$-value calculations for the \test{best friend test} and the \test{friends test}.


\subsection{Symmetric attention matrix $\mathcal{A}$}
% \textcolor{blue}{\textit{The matrix shows relation tags vs collection and it is not quadratic. How the non-quadratic matrix can be symmetric?}}
It may happen that the number of tags $n$ coincides with the number of collections $k$, $n = k$. The particular case of the symmetric (by construction) attention matrix $\mathcal{A}$ requires a more detailed investigation.

The $p$-value calculation relies on the assumption of the independence of attention values in different collections. If the attention matrix $\mathcal{A}$ is symmetric by the nature of the underlying bipartite graph, this assumption does not hold. Thus, the theoretical inference presented in Section~\ref{sec:theory} is not valid anymore.

Still, the numerical procedure works. We show this in more detail in Supplement~\ref{seq:symmetric_a}. Of note, sometimes we know that all the diagonal elements are $0$ by construction. In this case, both tests use 
\[
p(w) = (1-w)^{k-1}, ~~\text{cf. equation \eqref{eq:pw}.}
\]

\subsection{Multiple testing}
\label{sec:multimurkers}

All the tests we formulated here are not corrected for the multiplicity of hypotheses. Namely, they work directly if we \textit{a-priori} know what tag $t_i$ and what size $l$ of friends set we run the test for. 

In practice, the tests are run for each tag or even for each tag and the friend set population. Two important observations follow.

\paragraph*{Multiplicity correction} 
To run the \test{best friend test} on all the $n$ tags, we 
calculate $n$ $p$-values. To run the \test{friends test}, we calculate $n \cdot k$ $p$-values. However, all the tests rely on the ranking of the elements inside the same attention matrix $\mathcal{A}$. Thus, the assumption of test independence does not hold. In this case, the standard Bonferroni correction on the set of corresponding $p$-values is possibly too strong (\cite{cabin2000bonferroni}). However, some correction is still necessary. We leave it to the scope of the particular application.

\paragraph*{Multimarkers} Note that after the multiple hypothesis correction, a collection may be the best friend (or an element of the true friends set) for more than one tag. The set of tags is thus a multimarker for the collection. In practice, a multimarker tags (selects) a collection more specific than each of its elements.

\subsection{The code availability}

The software that implements the \text{best friend test} and \text{friends test} is available as an \textsf{R} package \textsf{best.friends} at 
\url{https://github.com/favorov/best.friends}. The vignette of the package shows simple usecases. 

\section{Discussion}

In this manuscript, we develop a method and software to detect noteworthy edges in a weighted bipartite graph. We suppose all edges of the graph to be co-directed. The graph models a directed relation (referred to as attention) from the vertices of one part (collections) to the vertices of another part (tags).

Essentially, the method consists of two steps. First, we use a double-ranking approach to find the putative friends. Then, to validate the friendship hypothesis, we perform a novel statistical test that is distribution-free.

Along with the single collection procedure (\test{best friend test}), we suggest its extension for a subset of collections (\test{friends test}).

The \text{best friend test} is a particular case of the \test{friends test}, but we consider it separately in the software and, hence, in the methods. Namely, \test{friends test} has higher computational costs and it requires multiplicity correction even for one tag (Section \ref{sec:multimurkers}). In many cases, the \text{best friend test} is enough for practical applications. 

Although the problem looks abstract, its solution has numerous straightforward applications. For instance, the detection of the marker genes \cite{stein-obrien_patternmarkers_2017} for expression patterns critically simplifies the biological interpretation of the results of transcription matrix factorization (\cite{Stein_2018,Fertig_2016}). Here, the genes are tags and the patterns are collections. If a pattern is a friend of a gene (see Section \ref{sec:method}), the gene is the marker of the pattern.

However, the theoretical result is limited to the case of an asymmetric attention matrix. If the matrix is symmetric by the design, the null hypothesis does not hold. However, the computational experiment \textcolor{red}{(see Supplement)} shows that the independence proposition \textcolor{red}{can be used}. Thus, the method applies to the analysis of, e.g. distance matrices. 

The first possible area of application is feature selection. By identification of markers, instead of all tags, we can use a relatively small subset for further analysis. Moreover, the identification of friend-marker pairs helps to remove non-specific connections from a graph. 

Second, the proposed method is useful for efficient clustering of a set of selected features. Also, the friendship concept provides a new similarity measure that possibly generates more interpretable clustering, than the clustering with $\mathcal{A}$ being a similarity measure.

Another possible direction is knowledge transfer: if we know something new about Augustus, we know something new about his friends. 


% \textcolor{blue}{+ Friedman test }
% %\url{https://en.wikipedia.org/wiki/Friedman_test}


\section{Conflict of interest}
The authors declare no conflict of interest.

\section{Acknowledgements}
AF acknowledges support by National Institutes of Health (NIH) P30CA006973 and 1U01CA253403-01.
Thanks to Daniel Shu and Caedmon Haas for the translation of the motto to gold Latin. 

\bibliography{gene-best-friends}

\bibliographystyle{splncs03}
%\begin{subappendices}

\newcommand{\beginsupplement}{%
        \setcounter{table}{0}
        \renewcommand{\thetable}{S\arabic{table}}%
        \setcounter{figure}{0}
        \renewcommand{\thefigure}{S\arabic{figure}}
        \setcounter{equation}{0}
        \renewcommand{\theequation}{S\arabic{equation}}%
     }

\newpage
\section*{Supplement}
\beginsupplement
\subsection{Estimation of volumes and p-value}
\label{seq:inference}
%\renewcommand{\thesection}{\Alph{section}}%
% or try \arabic{section}
First, we recall that for any $m > 0$,
\begin{eqnarray}
&\displaystyle \int \limits_a^b \left(b-x\right)^{m-1}dx=
\displaystyle \frac{\left(b-a\right)^m}{m} . \label{eq:intab}
\end{eqnarray} 
%\begin{eqnarray}
%&\displaystyle \int_a^1\left(1-x\right)^{m-1}dx=
%-\displaystyle \int_0^{1-a}v^{n-1}dv=
%\displaystyle \frac{\left(1-a\right)^m}{m}  \label{eq:right}
%\end{eqnarray} 
%\begin{eqnarray}
%&\displaystyle \int_0^b\left(b-x\right)^{m-1}dx=\displaystyle \frac{b^m}{m}  \label{eq:left}
%\end{eqnarray}

% Applying \eqref{eq:intab} recursively from $k$ to $1$, we get
% \begin{align}
% V_k & = \displaystyle \int\limits_0^1\int\limits_{u_1}^1\int\limits_{u_2}^1\int\limits_{u_3}^1...\int\limits_{u_{k-1}}^1 du_k...du_4 du_3 du_2 du_1  \nonumber \\ 
% & =\displaystyle \frac{1}{(k-3)!}\int\limits_0^1\int\limits_{u_1}^1\int\limits_{u_2}^1 \left( 1-u_3 \right)^{k-3}du_3 du_2 du_1  \nonumber \\
% & =\displaystyle \frac{1}{(k-2)!}\int_0^1\int\limits_{u_1}^1\left( 1-u_2 \right)^{k-2} du_2 du_1  \nonumber \\
% & = \displaystyle \frac{1}{(k-1)!} \int\limits_0^1\left( 1-u_1 \right)^{k-1} du_1 = \frac{1}{k!}. \label{eq:volume}
% \end{align}

First, we consider the case for $l = 1$,
\begin{align}
 V_{1}(w) &=  \displaystyle \displaystyle  \int\limits_0^{1-w}\int\limits_{{u_1}+w}^1\int\limits_{u_2}^1\int\limits_{u_3}^1...\int\limits_{u_{k-1}}^1 du_k...du_2 du_1 \nonumber \\ 
& = \displaystyle \frac{1}{(k-2)!}\int\limits_0^{1-w}\int\limits_{{u_1}+w}^1\left( 1-u_2 \right)^{k-2} du_2 du_1 \nonumber \\
& = \displaystyle \frac{1}{k!} \int\limits_0^{1-w}\left( 1-w-u_1 \right)^{k-1} du_1 = \frac{(1-w)^k}{k!}. \label{eq:p_1}
\end{align}

Now we consider $l\ge 2$,
\begin{align}
V_{l}(w) & = \displaystyle \int\limits_0^{1-w}\int\limits_{{u_1}}^{1-w}...\int\limits_{u_{l-1}}^{1-w}\int\limits_{u_l+w}^1...\int\limits_{u_{k-1}}^1 du_k... du_1  \nonumber \\ 
& =  \displaystyle \frac{1}{(k-l-1)!}\displaystyle \int\limits_0^{1-w}\int\limits_{{u_1}}^{1-w}...\int\limits_{u_{l-1}}^{1-w}\int\limits_{u_l+w}^1 \left( 1-u_{l+1} \right)^{k-l-1} du_{l+1}...du_1   \nonumber \\
&=  \displaystyle \frac{1}{(k-l)!}\displaystyle \int\limits_0^{1-w}\int\limits_{{u_1}}^{1-w}...\int\limits_{u_{l-1}}^{1-w} \left( 1-t-u_{l} \right)^{k-l} du_l...du_1   \nonumber \\
& = \displaystyle \frac{1}{(k-l+1)!}\displaystyle \int\limits_0^{1-w}\int\limits_{{u_1}}^{1-w}...\int\limits_{u_{l-2}}^{1-w} \left( 1-w-u_{l-1} \right)^{k-l+1} du_{l-1}...du_1   \nonumber \\
& =  \displaystyle \frac{1}{(k-1)!} \int\limits_0^{1-w}\left( 1-w-u_1 \right)^{k-1} du_1 = \frac{(1-w)^k}{k!} \label{eq:p_k}
\end{align}

\subsection{Symmetric $\mathcal{A}$}
\label{seq:symmetric_a} 

% \begin{eqnarray}
% & p(w) =  \displaystyle \frac{1}{Q}\displaystyle \int_0^{1-w}\int_{{u_1}+w}^1\int_{u_2}^1\int_{u_3}^1...\int_{u_{k-1}}^1 du_k...du_2 du_1 =  \nonumber \\ 
% %&\displaystyle \frac{n!}{(n-3)!}\int_0^{1-t}\int_{{u_1}+t}^1\int_{u_2}^1 \left( 1-u_3 \right)^{n-3}du_3 du_2 du_1 =  \nonumber \\
% &\displaystyle \frac{k!}{(k-2)!}\int_0^{1-w}\int_{{u_1}+w}^1\left( 1-u_2 \right)^{k-2} du_2 du_1 =  \nonumber \\
% &\displaystyle k \int_0^{1-w}\left( 1-w-u_1 \right)^{k-1} du_1 = (1-w)^k \label{eq:p_1}
% \end{eqnarray}


% \begin{eqnarray}
% & p_l(w) = \displaystyle \frac{1}{Q}\displaystyle \int_0^{1-w}\int_{{u_1}}^{1-w}...\int_{u_{l-1}}^{1-w}\int_{u_l+w}^1...\int_{u_{k-1}}^1 du_k... du_1 =  \nonumber \\ 
% & \displaystyle \frac{k!}{(k-l-1)!}\displaystyle \int_0^{1-w}\int_{{u_1}}^{1-w}...\int_{u_{l-1}}^{1-w}\int_{u_l+w}^1 \left( 1-u_{l+1} \right)^{k-l-1} du_{l+1}...du_1 =  \nonumber \\
% & \displaystyle \frac{k!}{(k-l)!}\displaystyle \int_0^{1-w}\int_{{u_1}}^{1-w}...\int_{u_{l-1}}^{1-w} \left( 1-t-u_{l} \right)^{k-l} du_l...du_1 =  \nonumber \\
% & \displaystyle \frac{k!}{(k-l+1)!}\displaystyle \int_0^{1-w}\int_{{u_1}}^{1-w}...\int_{u_{l-2}}^{1-w} \left( 1-w-u_{l-1} \right)^{k-l+1} du_{l-1}...du_1 =  \nonumber \\
% & \displaystyle k \int_0^{1-w}\left( 1-w-u_1 \right)^{k-1} du_1 = (1-w)^k \label{eq:p_k}
% \end{eqnarray}
%\end{subappendices}
\begin{comment}
\subsection{ex-abstract}
Suppose we have set of {\tag}s $\T$ and a set of {\collection}s $\C$ of this {\tag}s with some fuzzy membership, e.g. there is a numeric measure of how an {\tag} is represented in a {\collection} $A(\tl,\cl)$ for each pair $(\tl,\cl):\tl \in \T, \cl \in \C$. The higher the $A(\tl,\cl)$ value is, the more the {\tag} $\tl$ is involved in the {\collection} $\cl$. The absence of the {\tag} in the {\collection} is shown by the value that is minimal for the {\collection}.

An example that is easy to think about is: genes are {\tag}s and their groups under regulation by same transcription factors (TF’s) form {\collection}s, and A shows the strength of the regulation. We look at a gene and we want to know whether a TF is its friend, i.e. whether the TF specifically prefers (regulates) the gene. The na\"ive idea is to look for a TF that the gene is the most sensitive for. Still, it’s possible that this TF is the strongest for the most of the genes. Sometimes, it is what we want to find, but now we want to answer other questions, namely, what TF is the most specific factor for the gene and is the specificity enough to say that is does not look like a random outcome?

Sometimes, the {\collection}s are in in one-to-one relation with the {\tag}s, e.g. each {\collection} is a set of {\tag}s, which are neighbours of {\tag} in some graph. For this example, A is a weighted adjacency matrix of this graph. The friendship terminology emerges naturally from this case. The friendship relation itself is asymmetric: a friend cares about Augustus, while Augustus does not.
\end{comment}




\end{document}


